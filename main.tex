\documentclass[11pt,a4paper]{moderncv}

\moderncvstyle[alignment=left]{banking}
\moderncvcolor{black}

\nopagenumbers{}

\usepackage[english]{babel}
\usepackage[utf8]{inputenc}
\usepackage[T1]{fontenc}
\usepackage[sc]{mathpazo}
\usepackage{microtype}
\usepackage{enumitem}

\usepackage[scale=0.85]{geometry}

\recomputelengths%

% Personal --------------------------------------------------------------------
\name{Rodrigo}{\texorpdfstring{Marques\vspace*{.5em}}{Marques}}% add some whitespace between name and contacts
 
\phone[mobile]{+351~914~398~959}%
\email{rodrigo.marques@fc.up.pt}%
\homepage{https://marques-rodrigo.github.io/}
\social[linkedin]{rodrigo-marques2}%
\social[github]{marques-rodrigo}%
\social[orcid]{0000-0003-2492-0197}%

\begin{document}

\makecvtitle%

\vspace*{-3em}% reduce blank space between cv title and cv body

% ---------------------------------------------------------------------------------------------------------------------

\section{Research Interests}

Programming Languages, Type Systems, Semantics, Substructural Logics.

% ---------------------------------------------------------------------------------------------------------------------

\section{Education}

\begin{description}[labelwidth=60pt,align=right,leftmargin=!]
    \item[\normalfont{\emph{2019 -- 2023}}] \textbf{MSc Computer Science (Reliable Computing), University of Porto}
    \item[\normalfont{\emph{2016 -- 2019}}] \textbf{BSc Computer Science, University of Porto}
\end{description}

% ---------------------------------------------------------------------------------------------------------------------

\section{Employment}

\begin{description}[labelwidth=60pt,align=right,leftmargin=!]
    \item[\normalfont{\emph{2023 -- now}}] \textbf{Invited Assistant (University Teacher), University of Porto}
    \\ \small{Lab instructor for courses in programming and in theoretical computer science.}
    \item[\normalfont{\emph{2021 -- 2022}}] \textbf{Gameplay Programmer, NORTHWIND}
    \\ \small{Developed an ocean simulation and sailing mechanics as an extension to the game.}
    \\ \small{Introduced version control. Integrated game with PlayFab analytics.}
\end{description}

% ---------------------------------------------------------------------------------------------------------------------

\section{Teaching}

\begin{description}[labelwidth=60pt,align=right,leftmargin=!]
    \item[\normalfont{\emph{Fall 2024}}] \textbf{Functional and Logic Programming} \emph{(Haskell, Prolog)} (L.EIC024)
    \item[\normalfont{\emph{Spring 2024}}] \textbf{Functional Programming} \emph{(Haskell)} (CC1005)\\
                                         \textbf{Programming} \emph{(C++)} (L.EIC009)                                         
    \item[\normalfont{\emph{Fall 2023}}] \textbf{Introduction to Programming} \emph{(Python)} (CC1024)\\
                                         \textbf{Programming Fundamentals} \emph{(Python)} (L.EIC003)
\end{description}

% ---------------------------------------------------------------------------------------------------------------------

\section{Publications}
% \subsection*{Conference Papers}
\begin{description}[labelwidth=60pt,align=right,leftmargin=!]
    \item[\normalfont{\emph{Haskell 2024}}] 
        \textbf{Haskelite: A Tracing Interpreter Based on a Pattern-Matching Calculus}
        \\ \emph{Pedro Vasconcelos, \textbf{Rodrigo Marques}}
        \\ \emph{Proceedings of the 17th ACM SIGPLAN International Haskell Symposium}    

    % \item[\normalfont{\emph{IFL 2023}}]
    %     \textbf{A Lazy Abstract Machine Based on a Pattern-Matching Calculus}
    %     \\ \emph{Pedro Vasconcelos, \textbf{Rodrigo Marques}}
    %     \\ Presented at: Symposium on Implementation and Application of Functional Languages
\end{description}

\subsection*{Dissertation}

\begin{description}[labelwidth=60pt,align=right,leftmargin=!]
    \item[\normalfont{\emph{MSc}}] \textbf{Subtyping: Study and Implementation}
                                \\ \emph{Studied type inference for the combination of algebraic subtyping, parametric polymorphism, and row and presence polymorphism.}
\end{description}

\section{Presentations}

% \subsection*{Extended Abstracts}
\begin{description}[labelwidth=60pt,align=right,leftmargin=!]
    \item[\normalfont{\emph{ML 2022}}]
        \textbf{Towards Algebraic Subtyping for Extensible Records}
        \\ \emph{\textbf{Rodrigo Marques}, Mário Florido, Pedro Vasconcelos}                                
        \\ Extended abstract presented at the ML Workshop at the 27th ACM SIGPLAN International Conference on Functional Programming
\end{description}

% ---------------------------------------------------------------------------------------------------------------------

\section{Scholarships and Travel Grants}

\begin{description}[labelwidth=60pt,align=right,leftmargin=!]
    \item[\normalfont{\emph{2024}}]
           Proof Society International School
        \\ International School on Rewriting 
        \\ Oregon Programming Languages Summer School
\end{description}

\begin{description}[labelwidth=60pt,align=right,leftmargin=!]
    \item[\normalfont{\emph{2023}}]
        EuroProofNet Summer School on Verification Technology, Systems \& Applications
\end{description}

\end{document}